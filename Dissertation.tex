%% LaTeX template for BSc Computing for Games final year project dissertations
%% by Edward Powley
%% Games Academy, Falmouth University, UK

%% Based on:
%% bare_jrnl.tex
%% V1.4b
%% 2015/08/26
%% by Michael Shell
%% see http://www.michaelshell.org/
%% for current contact information.
%%
%% This is a skeleton file demonstrating the use of IEEEtran.cls
%% (requires IEEEtran.cls version 1.8b or later) with an IEEE
%% journal paper.
%%
%% Support sites:
%% http://www.michaelshell.org/tex/ieeetran/
%% http://www.ctan.org/pkg/ieeetran
%% and
%% http://www.ieee.org/

%%*************************************************************************
%% Legal Notice:
%% This code is offered as-is without any warranty either expressed or
%% implied; without even the implied warranty of MERCHANTABILITY or
%% FITNESS FOR A PARTICULAR PURPOSE! 
%% User assumes all risk.
%% In no event shall the IEEE or any contributor to this code be liable for
%% any damages or losses, including, but not limited to, incidental,
%% consequential, or any other damages, resulting from the use or misuse
%% of any information contained here.
%%
%% All comments are the opinions of their respective authors and are not
%% necessarily endorsed by the IEEE.
%%
%% This work is distributed under the LaTeX Project Public License (LPPL)
%% ( http://www.latex-project.org/ ) version 1.3, and may be freely used,
%% distributed and modified. A copy of the LPPL, version 1.3, is included
%% in the base LaTeX documentation of all distributions of LaTeX released
%% 2003/12/01 or later.
%% Retain all contribution notices and credits.
%% ** Modified files should be clearly indicated as such, including  **
%% ** renaming them and changing author support contact information. **
%%*************************************************************************


\documentclass[journal]{IEEEtran}

\usepackage{graphicx}
% Insert additional usepackage commands here

\begin{document}
%
% paper title
% Titles are generally capitalized except for words such as a, an, and, as,
% at, but, by, for, in, nor, of, on, or, the, to and up, which are usually
% not capitalized unless they are the first or last word of the title.
% Linebreaks \\ can be used within to get better formatting as desired.
% Do not put math or special symbols in the title.
\title{Your title here}
%
%
% author name
\author{Tomas Mazurkevic}

% The paper headers -- please do not change these, but uncomment one of them as appropriate
% Uncomment this one for COMP320
\markboth{COMP320: Research Review and Proposal}{COMP320: Research Review and Proposal}
% Uncomment this one for COMP360
% \markboth{COMP360: Dissertation}{COMP360: Dissertation}

% make the title area
\maketitle

% As a general rule, do not put math, special symbols or citations
% in the abstract or keywords.
\begin{abstract}
The abstract goes here.
\end{abstract}

\section{Introduction}
% The very first letter is a 2 line initial drop letter followed
% by the rest of the first word in caps.
% 
% form to use if the first word consists of a single letter:
% \IEEEPARstart{A}{demo} file is ....
% 
% form to use if you need the single drop letter followed by
% normal text (unknown if ever used by the IEEE):
% \IEEEPARstart{A}{}demo file is ....
% 
% Some journals put the first two words in caps:
% \IEEEPARstart{T}{his demo} file is ....
% 
% Here we have the typical use of a "T" for an initial drop letter
% and "HIS" in caps to complete the first word.
\IEEEPARstart{A}{rtificial} intelligence (AI) in video games is very crucial when it's used as one of the main mechanics. Intelligent non-player characters (NPCs) are extremely hard to make due to the nature of it's complexity. The more complex it is the more outcomes developers will have to consider and the more problems will occur during development. 

Explain curiosity - human behaviour often depends on person's intensions to explore. Why apply to AI? The aim of this paper is not to create curiosity-driven AI in Unreal Engine 4 (UE4), but to implement and test a particular behaviour and see how people react. Whether they like it or not, whether they think it behaves smart and etc. The aim of this implementation is to find a small nuance that can make the AI feel and act smarter.

\section{Related Work} %~134
The main focus of this paper is to (demonstrate) talk about a potentially valuable AI behavior that is not complex yet efficient in demonstrating (implementing) smart AI. In Section III different AI implementation methods will be discussed, both complex and simple. Section IV will present (introduce) human's curiosity and discuss its importance towards one's willingness to discover and possibly relevant behavior implementations to make smart AI. The testing material will be discussed in Section V, which will also include an explanation and use of Turing's test. Methodology and hypotheses will be reviewed in sections VI and VII respectively. The behavior implementation will be made in Unreal Engine 4 (UE4) using our team's game for easier testing purpose as there will be enough assets to create a unique world which AI could be placed on.
\begin{itemize}
	\item Explain the main focus of this paper
	\item Explain the testing subject I'm gonna use
	\item Briefly explain next sections
	\item Will be using my third year team project game to test the implementation on.
\end{itemize}

\section{AI Evolution}
Different methods that are used in AI learning - reinforcement, evolution-based, q-learning, etc. 
\begin{itemize}
	\item Why is it complex and important?
	\item What does it help to achieve? 
	\item Talk about Halo 2 AI (and maybe TF2 bots - research more)
	\item Talk about how simple implementations can be much more efficient than complex learning methods
\end{itemize}

\section{Curiosity}
\begin{itemize}
	\item Why do curiosity?
	\item Examples of curiosity-driven behaviours
	\item How will it make AI look smarter?
	\item Challenges that occur trying to implement such behaviour
	\item Explain why curiousity is one of the main drivers that lead human to explore/do something
\end{itemize}

\section{Turing's Test}
\begin{itemize}
	\item Explain Turing's test
	\item Why use this testing method?
	\item Explain how I'm gonna test my implementation
	\item Briefly go over what question I'm going to use to get the participants' data
\end{itemize}

\section{Methodology}
\begin{itemize}
	\item How the hypotheses will be tested?
	\item Include potential questions
	\item Maybe include an example of similar test?
\end{itemize}

\section{Hypotheses}
\begin{itemize}
	\item The AI will feel smarter
	\item The participants won't notice the difference
	\item The participants won't distinguish human's behaviour and AI (which is good)
\end{itemize}

\section{Conclusion}
The conclusion goes here.

% references section

\bibliographystyle{IEEEtran}
\bibliography{references}

% Appendices

\appendices
\section{First appendix}
Appendices are optional. Delete or comment out this part if you do not need them.

% that's all folks
\end{document}
