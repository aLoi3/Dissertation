%% LaTeX template for BSc Computing for Games final year project dissertations
%% by Edward Powley
%% Games Academy, Falmouth University, UK

%% Based on:
%% bare_jrnl.tex
%% V1.4b
%% 2015/08/26
%% by Michael Shell
%% see http://www.michaelshell.org/
%% for current contact information.
%%
%% This is a skeleton file demonstrating the use of IEEEtran.cls
%% (requires IEEEtran.cls version 1.8b or later) with an IEEE
%% journal paper.
%%
%% Support sites:
%% http://www.michaelshell.org/tex/ieeetran/
%% http://www.ctan.org/pkg/ieeetran
%% and
%% http://www.ieee.org/

%%*************************************************************************
%% Legal Notice:
%% This code is offered as-is without any warranty either expressed or
%% implied; without even the implied warranty of MERCHANTABILITY or
%% FITNESS FOR A PARTICULAR PURPOSE! 
%% User assumes all risk.
%% In no event shall the IEEE or any contributor to this code be liable for
%% any damages or losses, including, but not limited to, incidental,
%% consequential, or any other damages, resulting from the use or misuse
%% of any information contained here.
%%
%% All comments are the opinions of their respective authors and are not
%% necessarily endorsed by the IEEE.
%%
%% This work is distributed under the LaTeX Project Public License (LPPL)
%% ( http://www.latex-project.org/ ) version 1.3, and may be freely used,
%% distributed and modified. A copy of the LPPL, version 1.3, is included
%% in the base LaTeX documentation of all distributions of LaTeX released
%% 2003/12/01 or later.
%% Retain all contribution notices and credits.
%% ** Modified files should be clearly indicated as such, including  **
%% ** renaming them and changing author support contact information. **
%%*************************************************************************


\documentclass[journal]{IEEEtran}

\usepackage{graphicx}
% Insert additional usepackage commands here

\begin{document}
%
% paper title
% Titles are generally capitalized except for words such as a, an, and, as,
% at, but, by, for, in, nor, of, on, or, the, to and up, which are usually
% not capitalized unless they are the first or last word of the title.
% Linebreaks \\ can be used within to get better formatting as desired.
% Do not put math or special symbols in the title.
\title{Does Exploration Behaviour That is Based of (Human-Like) Curiosity Perceive AI Smartness?}
%
%
% author name
\author{Tomas Mazurkevic}

% The paper headers -- please do not change these, but uncomment one of them as appropriate
% Uncomment this one for COMP320
\markboth{COMP320: Research Review and Proposal}{COMP320: Research Review and Proposal}
% Uncomment this one for COMP360
% \markboth{COMP360: Dissertation}{COMP360: Dissertation}

% make the title area
\maketitle

% As a general rule, do not put math, special symbols or citations
% in the abstract or keywords.
\begin{abstract}
The abstract goes here.
\end{abstract}

\section{Introduction}
% The very first letter is a 2 line initial drop letter followed
% by the rest of the first word in caps.
% 
% form to use if the first word consists of a single letter:
% \IEEEPARstart{A}{demo} file is ....
% 
% form to use if you need the single drop letter followed by
% normal text (unknown if ever used by the IEEE):
% \IEEEPARstart{A}{}demo file is ....
% 
% Some journals put the first two words in caps:
% \IEEEPARstart{T}{his demo} file is ....
% 
% Here we have the typical use of a "T" for an initial drop letter
% and "HIS" in caps to complete the first word.
\IEEEPARstart{A}{rtificial} intelligence (AI) in video games is very crucial when it's used as one of the main mechanics. Intelligent non-player characters (NPCs) are extremely hard to make due to the nature of it's complexity. The more complex it is the more outcomes developers will have to consider and the more problems will occur during development. 

Explain curiosity - human behaviour often depends on person's intentions to explore. Why apply to AI? The aim of this paper is not to create curiosity-driven AI in Unreal Engine 4 (UE4), but to implement and test a particular behaviour and see how people react. Whether they like it or not, whether they think it behaves smart and etc. The aim of this implementation is to find a small nuance that can make the AI feel and act smarter.

Throughout the last 30 years researchers are trying to create intelligent AI using different methods and algorithms. Recent achievements in such are are really worthy of respect, for example, AlphaGo or OpenAI (citation required). AI in games, however, is used to create non-player characters (NPCs), which can be enemies, merchants, companions and a lot more. Some of the very complex algorithms, such as reinforcement learning (RL), deep Q-learning, curiosity-driven, evolution-based (add more or change) are being used in old games such as Atari as a test. However, due to specificity of games, it is not always the best solution. Depending on game's design, very simple methods can be used instead, or less complex than previously mentioned ones. As an example, Monte Carlo Tree Search (MCTS) became a common one to use after the success of the game F.E.A.R (i guess?) or Finite State Machine (FSM), which is also a very common one these days and is being used in most known games (citation required). Another widely spread technic is behaviour tree, which I'm going to use for this paper.

\section{Related Work} %~134
The main focus of this paper is to (demonstrate) talk about a potentially valuable AI behaviour that is not complex yet efficient in demonstrating (implementing) smart AI. In Section III different AI implementation methods will be discussed, both complex and simple. Section IV will present (introduce) human's curiosity and discuss its importance towards one's willingness to discover and possibly relevant behaviour implementations to make smart AI. The testing material will be discussed in Section V, which will also include an explanation and use of Turing's test. Methodology and hypotheses will be reviewed in sections VI and VII respectively. The behaviour implementation will be made in Unreal Engine 4 (UE4) using our team's game for easier testing purpose as there will be enough assets to create a unique world which AI could be placed on.
\begin{itemize}
	\item Explain the main focus of this paper
	\item Explain the testing subject I'm gonna use
	\item Briefly explain next sections
	\item Will be using my third year team project game to test the implementation on.
\end{itemize}

\section{AI Evolution or AI Importance in Games}
Different methods that are used in AI learning - reinforcement, evolution-based, q-learning, curiosity-driven, etc. 
\begin{itemize}
	\item Why is it complex and important?
	\item What does it help to achieve? 
	\item Talk about Halo 2 AI (and maybe TF2 bots - research more)
	\item Talk about how simple implementations can be much more efficient than complex learning methods
\end{itemize}

Artificial intelligence has a big place in the modern world. Robots are being created to ease human's life, machines that can beat best players in the world in most complex games such as Go (cite AlphaGo). But AI in games is not created to overcome player's abilities. It's there to enhance player's experience, to make the game more engaging and fun to play. Therefore, other difficult problems are being solved, which may not fit well with AI used in real life, but they are still very valuable. Solving AI problems using games has become pretty common nowadays and is used for many areas as a testing purpose (possible citation here).

AI for games is mainly developed to increase player's engagement and satisfaction aspects (maybe more), give the ability to set the difficulty level, create a relationship between the player and NPCs and much more (get more potential citation here). However, some games use it differently. Some games want to present players with feelings such as frustration, fear, anger (and more/possible citation). It is designers' desire to create such feelings depending on the game itself, which proves the flexibility of AI for games.

Artificial intelligence is very commonly used to create feelings alongside an atmosphere of the game. For example, in Alien: Isolation, the alien enemy is the main threat and making the player fear it to create a horrifying experience. However, atmospehere around it also plays a huge role. Moreover, specifically in this game, the AI is very complex and smart. In fact, it has two brains (citation needed). First one playing the role of what the Alien actually sees and hears and the other is having a memory of the player's position at all times, in case the player hides for a long time to create constant progression and prevent the player from playing too carefully.

Another excellent example of smart and flexible AI is Halo 2 developed by Bungie. Enemies in that game have roles of some kind: leaders, recruits and minions. These are not exact names for it, but to make it easier to distinguish while talking about. Minions are some small creatures, which are not very dangerous, because usually they walk alone or in very small groups and have little health and damage. However, they can be warned about the player or warn others of the player's position, which makes them more dangerous than they look like. Recruits, on the other, hand are much more dangerous not only because of damage and health, but also because they are lead by leaders, which they can communicate with. Moreover, leaders control several recruits and can communicate with other leaders giving them information about the player and potentailly requesting help or merging, depending on each goals. The most important stuff with this is that all enemies are somewhat connected with each other, which gives them the ability to communicate and give each other information necessary to hunt the player down.

AI importance in games
\begin{itemize}
	\item What makes AI important in games?
	\item What does it help to accomplish in games? Satisfaction, level of difficulty, engagement, relationship between the player and NPCs
	\item Has a big research area - making AI interesting and solving complex problems, which can be applied to real world problems with robots
	\item Potentially talk about Halo 2 AI, which made the game very engaging and fun to play - as an example
\end{itemize}

\section{Curiosity}
Curiosity is a driving force for human's exploration, which consists of exploration, investigation and learning behaviours (cite "Curiosity: From Psychology to Computation"). It makes people chase for knowledge and investigate anything new and potentially valuable (for any reason). Moreover, curiosity is beneficial for people on two levels: the individual and the social levels (cite "Kashdan" from "Curiosity: From Psychology to Computation"). The first one is represented as the "innate love of learning and of knowledge... without the lure of any profit" (cite Loewenstein). The social level, on the other hand, is presented as "an ingredient for enhancing personal relationships" (cite "Curiosity: From Psychology to Computation").
\begin{itemize}
	\item Why do curiosity?
	\item Examples of curiosity-driven behaviours
	\item How will it make AI look smarter?
	\item Challenges that occur trying to implement such behaviour
	\item Explain why curiosity is one of the main drivers that lead human to explore/do something
\end{itemize}

\section{Turing's Test}
\begin{itemize}
	\item Explain Turing's test
	\item Why use this testing method?
	\item Explain how I'm gonna test my implementation
	\item Briefly go over what question I'm going to use to get the participants' data
	\item Analyse critics regarding Turing's test and explain why I chose this method
\end{itemize}

\section{Hypotheses}
The aim of this paper is to implement a simple exploration behaviour and test it to see whether it makes the AI smarter. This concern is the first hypotheses of the paper. Previously mentioned testing, which requires participants to distinguish player's behaviour and AI's behaviour presents us with another hypotheses. This time we are looking if the AI's behaviour is human-like. Hypotheses 3 depends on the participants' answers - if they cannot distinguish between player and AI, it proves that the AI is human-like, which delivers positive outcome (for this paper).
\begin{itemize}
	\item The AI will feel smarter
	\item The participants won't notice the difference
	\item The participants won't distinguish human's behaviour and AI (which is good)
\end{itemize}

\section{Methodology}
\begin{itemize}
	\item How the hypotheses will be tested?
	\item Include potential questions
	\item Maybe include an example of similar test?
\end{itemize}

\section{Conclusion}
The conclusion goes here.

% references section

\bibliographystyle{IEEEtran}
\bibliography{references}

% Appendices

\appendices
\section{First appendix}
Appendices are optional. Delete or comment out this part if you do not need them.

% that's all folks
\end{document}
